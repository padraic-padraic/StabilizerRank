\documentclass{standalone}
\begin{document}
The pursuit of quantum computing is motivated by the promised `quantum speedup': quadratic or exponential reductions in the computational complexity over the best classical technqiues. However, the origins of this quantum advantage are not yet well understood. Quantum Parallelism~\cite{Deutsch1985}, unbounded multi-partite entanglement~\cite{Jozsa2003}, and most recently state-dependent contextuality~\cite{Howard2014} have all been identified as required for a speed-up over classical methods. 
\par
Developing techniques for classical simulation of quantum computation has emerged as an interesting method to try and constrain and study this speedup~\cite{Jozsa2003}. Any system that is clasically simulable cannot, by definition, exhibit quantum speedup, as we can simply classically simulate the quantum computation to obtain the result without developing a quantum device. The converse conjecture, however, is not necessarily true, as it depends on the framework used to study the quantum circuits.  
\par

\ifstandalone
\bibliography{../MresProject.bib}
\fi
\end{document}


















































