\documentclass{standalone}
\begin{document}

In this report, we have discussed the role of classical descriptions and simulations of quantum mechanics as a tool to try and understand the origins of quantum speedup. In particular, we focused on a novel simulation technique developed by Bravyi, Gosset, Smolin \& Smith. \\
This technique does not admit an efficient classical simulation of quantum computation, which is believed to be impossible. Instead, it is able to significantly reduce the classical complexity of simulating quantum circuits built using the Clifford+T gate set. This particular gate set lies at the heart of modern schemes for fault tolerant quantum computation.
\par
This reduction in complexity derives from the decomposition of the edge-type magic states used to realise T gates into a sum of $\chi$ stabilizer states, where $\chi$ is called the stabilizer rank. The stabilizer rank seems to be minimal for these edge-type states; combined with existing observations about the T-count of other quantum gates, and the stabilizer rank of arbitrary quantum states, we argue that computational savings achieved by the stabilizer rank method are related to the value of the state as a resource for fault tolerant computing.\\
Given this interpretation, we examined the behaviour of the stabilizer rank for alternative resource states: the class of face-type magic states, and resource states for gates in higher levels of the Clifford hierarchy. 
\par
The results seem to confirm that the edge-type magic states have the smallest stabilizer rank, and imply that there could be an advantage to building a fault tolerant scheme around alternative resource states, including gates from $\mathcal{C}_{n>3}$. Further work on adapting circuit synthesis techniques for different universal gate sets would be required to extend these results to give clear proposals for building quantum devices. 

\section*{Future Work}
These results, especially concerning $\mathcal{C}_{n>3}$ resource states, are preliminary, mostly due to computational constraints in finding the stabilizer rank decompositions with the brute-force search methods. The numerical techniques employed thus need further development. 
\par
The programme used to find the stabilizer states could be significantly improved in several ways. The simplest point is an implementation detail: rewriting the programme in \texttt{C} would allow finer memory management, reducing the footprint of the programme. \\
A more significant improvement would be to use an alternative representation of the stabilizer states, that does not require building the Pauli operators in each group and explicitly solving for the corresponding $+1$ eigenstate. This representation extends the use of binary subspaces on $\mathbb{F}_{2}^{n}$ to include binary quadratic functions~\cite{Dehaene2003}. It can then be shown that a stabilizer state is characterised by a linear subspace $\mathcal{L}\subseteq \mathbb{F}_{2}^{n}$, a quadratic function $q:\mathbb{F}_{2}^{n}\rightarrow\mathbb{F}_{2}$ and a `shift' vector $t\in\mathbb{F}_{2}^{n}$~\cite{Dehaene2003,Gross2007}. This is in fact the computational representation used by Bravyi \& Gosset in their implementation of the stabilizer rank method. Combined with a better method of generating these linear subspaces than brute-force search, this would allow stabilizer states to be generated much more efficiently. \\
Alternatively, given the good correspondence between the simulated annealing method and brute-force up to 4 qubits, switching to a simulated annealing search would allow the numerical analysis to be extended. The significant advantage here is that we only need to be able to generate $\chi$ unique, random $n$-qubit stabilizer states, rather than the full set. The other points in the phase space are then generated by applying Pauli operators to random states in the decomposition set $\tilde{\phi}$, as outlined in Algorithm~\ref{alg:randwalk}.
\par
Extending the technique used by Bravyi \& Gosset to bound the stabilizer rank beyond magic states would also allow us to find asymptotic bounds for the rank of the $\mathcal{C}_{n>3}$ states, to compliment further numerical work. 
\par
It would also be useful to understand the computational complexity of determining the stabilizer rank of a given state. This problem is evidently in $\NP$, as any candidate decomposition can be checked in polynomial time by building the projector and evaluating the overlap. \\
There is also an indication that the problem is $\NP$-hard. For example, finding the sparsest vector $x:Ax=b$ is known to be $\NP$-hard~\cite{ge2011note}. Setting $b$ as the computational basis representation and $A$ as the transformation between the computational and stabilizer state bases, as constructed in Section~\ref{sec:robustnessmeasure}, makes finding the sparsest solution $x$ equivalent to finding $\chi$. An alternative proof strategy could be based on the known $\NP$-hardness of finding sparse decompositions of doubly-stochastic matrices~\cite{Dufosse2016}.
\par
Finally, it would also be interesting to improve on the use of the \texttt{SL0} method to estimate the stabilizer rank. While the $\ell_{0}$ minimization problem is known to be $\NP$-hard, the related $\ell_{1}$ problem is in fact solvable exactly~\cite{Howard2016}. It is also known that under certain conditions, the global minima of the $\ell_{1}$ and $\ell_{0}$ problems coincide~\cite{Sawada2005}. Thus, it would be interesting to see if this heuristic can be used to solve for stabilizer rank by calculating the $0$-norm of solutions found using $\ell_{1}$ minimization techniques. 

\section*{Acknowledgements}
I would like to thank my supervisor Dr. Dan Browne for his discussions and input throughout the project, and Dr. Mark Howard for sharing his slides on the robustness measure for quantifying `magicness'. I would also like to thank the staff and students of the EPSRC CDT in Delivering Quantum Technologies for their training and support. 

\ifstandalone 
\bibliography{../MResProject.bib,../ManualEntries.bib}
\fi
\end{document}