\documentclass{standalone}
\begin{document}

The Stabilizer Rank algorithm is a new framework for clasically simulating quantum circuits built out of Clifford+T gates. In particular, it represents an 8-fold quadratic speedup over Aaronson \& Gottesman's \texttt{CHP} algorithm, with a complexity that scales as $\mathcal{O}(poly(n,t)2^{0.23 t})$ in $n$ qubits with $t$ T-gates. This method raises questions as to the value of the T-gate as a resource for Universal Quantum Computing. Here, we review classical simulations of quantum computation, and examine compelxity of different quantum resources in the Stabiliser Rank formalism. We argue that the stabilizer rank corresponds to the value of the state as a resource for quantum computation. 

\ifstandalone
\bibliography{../MresProject.bib}
\fi
\end{document}


